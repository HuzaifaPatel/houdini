\section{Testing Confinement}
\label{sec:testing-confinement}

\begin{table*}
  \caption{\todo{table caption}}
  \label{tab:classification}
  \centering \small
  \begin{tabular}{lllll}
    \toprule
    \bfseries No\@. & \bfseries Exploit & & \bfseries CVE No\@. & \bfseries Attack Vectors \normalfont{(See \Cref{fig:architecture})} \\
    \midrule
    1 & Mounted Docker Socket               & CITE & ---            & C, D, E, N \\
    2 & Mounted \texttt{/etc/passwd}        & CITE & ---            & C, D, G, K \\
    3 & Pipes Read-Only Overwrite           & CITE & CVE-2022-0847  & O, S \\
    4 & Cgroup Release Agent Code Execution & CITE & CVE-2022-0492  & M, O, R \\
    5 & Apache Path Traversal               & CITE & CVE-2021-42013 & D, I \\
    6 & \texttt{runc} Binary Overwrite      & CITE & CVE-2019-5736  & A, B, C, D, H, K \\
    \bottomrule
  \end{tabular}
\end{table*}

\todo{KEVIN: Explain some technical details for each exploit. Refer back to \Cref{tab:classification} as necessary.}

% https://unit42.paloaltonetworks.com/breaking-docker-via-runc-explaining-cve-2019-5736/
% https://nvd.nist.gov/vuln/detail/CVE-2019-5736
% exploits the container runtime runc [C, E?], the procfs filesystem [H, K], process IDs [P]
% RCE, elevation of privilege, DoS
\noindent\textbf{CVE-2019-5736.} This CVE\cite{addtheref} is a remote code execution vulnerability in the \texttt{runc} binary. The PoC exploits the way \texttt{runc} creates a process in the container called \texttt{runcInit} to run a specified command. The \texttt{procfs} file system contains a symlink to the binary being executed in \texttt{/proc/self/exe} which points to the \texttt{runc} executable on the host. A malicious container overwrites its own \texttt{/bin/sh} binary to the \texttt{/proc/self/exe} interpreter \texttt{\#!/proc/self/exe}. Once a user executes the overwritten \texttt{/bin/sh} binary, the interpreter calls the \texttt{runc} binary whose pid is then captured. Malicious code will then obtain the file descriptor from \texttt{/proc/runc-pid/exe} and overwrite the contents of the \texttt{runc} binary on the host.

% https://blog.qualys.com/vulnerabilities-threat-research/2021/10/27/apache-http-server-path-traversal-remote-code-execution-cve-2021-41773-cve-2021-42013
% https://nvd.nist.gov/vuln/detail/CVE-2021-42013
% exploits a specific vulnerable application in the container [D] through HTTP urls containing file traversal character sequences
% RCE
\noindent\textbf{CVE-2021-42013.} This is a remote code execution vulnerability in the Apache HTTP server container (versions 2.4.49 and 2.4.50). The exploit is a path traversal vulnerability that takes advantage of the server being unable to detect the path traversal characters ``../'' in a URL. This may occur when the second dot is replaced by its unicode representation ``\%2e''. By requesting a URL with several path traversal sequences and a binary to execute, contents of files can be retrieved from the container's local filesystem.

% https://dirtypipe.cm4all.com/
% https://nvd.nist.gov/vuln/detail/CVE-2022-0492
% exploits the cgroup file system [K, R?], cgroups [M], processes [I], special files that interact with the cgroups [H?], the way container filesystems are stored on the host [O, R?]
% RCE, elevation of privilege, information disclosure
\noindent\textbf{CVE-2022-0492.} A bug in the kernel's \detokenize{cgroup_release_agent_write cgroupv1} filesystem code can be exploited to enable a container escape. A file called \detokenize{release_agent} gets executed when a process in the cgroup terminates if \detokenize{notify_on_release} is enabled. The exploit involves mounting the cgroups filesystem on the host to the container and creating a directory which creates a new cgroup. Creating the \detokenize{notify_on_release} file in the new directory and writing a 1 to it tells the host to run the \detokenize{release_agent} file when a process in the custom cgroup terminates. Modifying the \detokenize{release_agent} file with a script that executes a malicious binary on the container from the host's filesystem allows the malicious container to successfully escape upon termination of a process in the custom cgroup.

% https://sysdig.com/blog/detecting-mitigating-cve-2022-0492-sysdig/
% https://nvd.nist.gov/vuln/detail/CVE-2022-0847
% exploits a vulnerability in the pipe.c code [S] that allows data within a spliced page cache to be overwritten if pipe is set up in a specific way
% RCE, elevation of privilege, DoS
\noindent\textbf{CVE-2022-0847.} This is an escalation of privilege Linux kernel vulnerability caused by a bug in the kernel's pipe.c code introduced in version 5.8. The exploit utilizes the pipe buffers flag \detokenize{PIPE_BUF_FLAG_CAN_MERGE} which allows for data to be overwritten in a page cache from a spliced file if the pipe buffer is prepared in a specific way. The flag gets set by filling the buffer with random data. The pipe buffer can then be emptied and then a file can be spliced into the pipe. Writing data into the pipe buffer now overwrites the data in the file. In order to escape a container with this exploit, the setup is similar to that of CVE-2019-5736 above where a malicious container waits for runc to execute in the container. The file descriptor is grabbed from the procfs and overwritten using the dirty pipe vulnerability.

We selected the exploits above based on several criteria. (1) The exploits must have a proof of concept readily available online; (2) the exploits must cover various components of the container deployment environment (see \Cref{fig:architecture}); and (3) the exploits must be high impact and severity defined by the CVS standards. The exploits chosen all fit within these criteria and can be tested repeatedly with different system configurations using \houdini.

The exploits are run in privileged and unprivileged mode with one of seccomp or apparmor securing the container. Most exploits are blocked successfully by using these security systems, but these security systems might not be configured or enabled by default. For example, kernels can come with apparmor on the system but it is disabled by default on boot and the service must be re-enabled by manipulating GRUB boot parameters. Using Docker info on a system lets a user know what container security mechanisms are available to the user. If apparmor is not installed on the system or the service is not enabled on boot through GRUB, then seccomp is usually the only security mechanism that can be deployed.

The default policy files for both apparmor and seccomp are used during testing. The default policy for apparmor most importantly restricts write permissions in the sensitive \texttt{/proc} and \texttt{/sys} file sysytems, as well as denying mounting operations. The default seccomp profile allows for all but 44 of the system calls to pass through and can allow more based on the container's capabilities. The system calls that are filtered out and restricted are focused around sensitive operations that can interact directly with the host kernel and potentially modify its behaviour (kernel modules, system time, reboot, etc.).

Tests were run on Ubuntu 22.04 with kernel version 5.15.0-50-generic. The docker engine is version 18.09.1 and runc version is 1.1.0+dev. For virtualization we used QEMU version 6.2.0 and buildroot version 2022.02.

We manually created a trick yaml file for each exploit outlining the environment and the commands for the host and container to perform. If required for the exploit, a new environment is created using the buildroot to generate a new virtual machine with a configurable linux kernel and filesystem. The versions for the kernel and specific packages that are required to run the exploit successfully, such as docker and runc, that are specified in the exploit yaml are passed to buildroot as parameters. Houdini is also included as a package and launches when the virtual system boots. QEMU runs the virtual environment and creates a virtual socket (vsock) connection between the host and the VM for status reporting and management. Once \houdini on the VM boots, a connection is made to \houdini running on the host. The trick is then sent via the vsock connection and parsed by the \houdini client on the VM and for the remainder of the trick, the VM is acting as the host. Once the trick finishes execution, the results are sent back to the host \houdini using the vsock connection and the VM is shut down.

The exploit CVE-2019-5736 is set up with a malicious container that runs a script on start to wait for the /bin/sh binary to execute and then in turn execute its own binary to overwrite the host runc. The vulnerable runc version being tested is version 1.0.0-rc6. On the host side, the user spins up the container and attempts to exec into the container using "docker exec -it ID sh". The exploit is considered a success if the runc binary on the host system has been compromised.

CVE-2021-42013 has a different threat model where the container image is vulnerable but not necessarily malicious. The exploit begins with a vulnerable apache httpd image being launched and then on the host side, the exploit binary is called using the container's IP address and a command as input. The exploit is considered a success if the command runs and data is received containing the response from the container.

Unlike the previous two exploits, CVE-2022-0492 does not utilize any malicious binaries. The exploit takes place with commands executed entirely within the container. A privileged container is launched and the cgroup rdma folder from the host cgroup filesystem is mounted in the container allowing for manipulation of the cgroup's release agent file. The exploit is considered successful if all commands execute successfully with no errors.

%%%%%%%%%%%%%%%%%%%% CVE-2022-0492 %%%%%%%%%%%%%%%%%%%%%%
% mkdir /tmp/mountest
% mount -t cgroup -o rdma cgroup /tmp/mountest
% mkdir /tmp/mountest/x
% echo 1 > /tmp/mountest/x/notify_on_release
% host_path=`sed -n 's/.*\perdir=\([^,]*\).*/\1/p' /etc/mtab`
% echo "$host_path/cmd" > /tmp/mountest/release_agent
% echo '#!/bin/sh' > /cmd
% echo "cat /etc/passwd > $host_path/output" >> /cmd
% chmod a+x /cmd
% sh -c "echo \$\$ > /tmp/mountest/x/cgroup.procs"
% cat /output

% display and explain results and how each test was ran

\begin{table*}
 \caption{\todo{table caption, replace success/failure text with glyphs. re-order, grouping unprivileged rows and privileged rows. }}
 \label{tab:results}
 \centering \small
 \begin{tabular}{lllll}
   \toprule
   \bfseries Security & \bfseries CVE-2019-5736 & \bfseries CVE-2021-42013 & \bfseries CVE-2022-0492 & \bfseries CVE-2022-0847 \normalfont{(See \Cref{fig:architecture})} \\
   \midrule
   unprivileged & success & success & failure & success \\
   privileged  & success & success & success & success \\
   unprivileged + apparmor & failure & success & failure & failure \\
   privileged + apparmor & failure & success & failure & failure\\
   unprivileged + seccomp & failure & success & failure & success\\
   privileged + seccomp & success & success &  success & success \\
   \bottomrule
 \end{tabular}
\end{table*}

\noindent\textbf{Why use Houdini.}

Container hardening techniques through confinement mechanisms require both knowledge of the application code and security best practices. Ensuring that their container environment is not vulnerable to known common exploits is a step that many developers will either save for the last part of their development cycle or forget about all together. Testing frameworks allow developers to save time by automatically running a suite of some tests in order to give the developer more insight into how their application behaves. Existing exploit testing frameworks are not designed with the goal of exposing problematic areas in a container's confinement policy or environment. \houdini allows the user to run a series of known vulnerabilities against their container environment to expose issues for the developer to quickly asses and act upon. The security mechanisms that are used to lockdown the container can easily be tested repeatedly throughout the development lifecycle without much overhead for the developer. \houdini also allows for the testing of new confinement mechanisms by academics and researchers with very little required customization to the trick files themselves. \houdini can test any part of the system for vulnerabilities including the kernel, runtime, and the container configuration which all contribute towards making the container vulnerable. This makes \houdini a powerful tool when assisting in the development of container confinement policies and environment testing.

\noindent\textbf{What makes running the exploit possible in apparmor.}

Apparmor and seccomp are integrated with docker to allow a user to further confine their resources and capabilities. When apparmor is enabled on a host kernel, docker will apply a default apparmor policy to a container. The policy name is docker default and it restricts: mounting, most cases of writing to /proc, and most cases of writing to /sys. This rather limited profile does a relatively good job of blocking common container escapes and privilege escalations. Issues mainly arise when a user tries to modify the existing policy, or develop their own without proper consideration. An apparmor profile is developed by specifying resources to allow or deny access to on each line. The profile must then be ran through the apparmor parser which checks for the proper syntax to be followed. Once the parser successfully reads and loads the profile, docker can use it as a value passed to their \detokenize{"--security-opts apparmor=PROFILE_NAME"} flag.

The apparmor parser will not check for whether a specified system path exists or not and if contradictory statements exist, it uses the statement that is more restrictive (i.e. if "mount" and "deny mount" exist in the same profile, it will always use "deny mount" no matter the declared order).

In order for CVE-2022-0492 to successfully operate with the apparmor security enabled on a privileged container, the only line that needs to be modified in the default profile is "deny mount". If a user were to change the "deny mount" to "mount" for either their own usage or as a simple mistake, then the exploit would be able to run successfully.

CVE-2019-5736 requires a little more modification to the docker default profile. Write access to \texttt{/proc/PID} must be allowed and full use of ptrace must be allowed. The exploit can successfully overwrite its own shell binary without any modifications to the default profile. If only write access to \texttt{/proc/PID} is given, the exploit can grab a hold of the runc process ID, but access to ptrace is required as it is used for process control by the system when modifying the \texttt{/proc/PID/exe} file. These two changes to the default apparmor profile allow CVE-2022-0492 to successfully overwrite the runc binary even without the privileged flag.

Using older versions of the container runtime RunC before version 1.0.0-rc8 allows for a different bypass of the apparmor confinement policy. Since apparmor is based off of path names, a malicious user can get around the default apparmor's policy that is locking down a specific path name by simply remounting it to a different path. An example of this is the procfs filesystem mounted on \texttt{/proc}. The Apparmor default profile locks down the procfs filesystem at the defined \texttt{/proc} location. By remounting the procfs filesystem to a different location, this bypasses the pathname based confinement that apparmor offers.

\noindent\textbf{What do the results tell us.}

\todo{docker default security profiles can be overly permissive. Why?}
\todo{seccomp by default allows almost 90\% of systemcalls}
\todo{apparmor allows all network capability}
\todo{default profiles are needed to be generic and more likely overly permissive}
\todo{generally do a good job catching existing exploits by locking down key parts}
\todo{Compromise between security and usability}
\todo{developing profiles can be difficult even with tools such as aa-easyprof}
\todo{testing confinement is made easy}


%%%%%%%%%%%%%%%%%%%%%%%%%%%% DEFAULT %%%%%%%%%%%%%%%%%%%%%%%%%%%%%%%%%%
% include <tunables/global>
%
% profile docker-default flags=(attach_disconnected,mediate_deleted) {
%
%   #include <abstractions/base>
%
%   ptrace peer=@{profile_name},
%
%   network,
%   capability,
%   file,
%   umount,
%
%   deny @{PROC}/* w,  # deny write for all files directly in /proc (not in a subdir)
%   # deny write to files not in /proc/<number>/** or /proc/sys/**
%   deny @{PROC}/{[^1-9],[^1-9][^0-9],[^1-9s][^0-9y][^0-9s],[^1-9][^0-9][^0-9][^0-9]*}/** w,
%   deny @{PROC}/sys/[^k]** w,  # deny /proc/sys except /proc/sys/k* (effectively /proc/sys/kernel)
%   deny @{PROC}/sys/kernel/{?,??,[^s][^h][^m]**} w,  # deny everything except shm* in /proc/sys/kernel/
%   deny @{PROC}/sysrq-trigger rwklx,
%   deny @{PROC}/kcore rwklx,
%   deny @{PROC}/mem rwklx,
%   deny @{PROC}/kmem rwklx,
%
%   deny mount,
%
%   deny /sys/[^f]*/** wklx,
%   deny /sys/f[^s]*/** wklx,
%   deny /sys/fs/[^c]*/** wklx,
%   deny /sys/fs/c[^g]*/** wklx,
%   deny /sys/fs/cg[^r]*/** wklx,
%   deny /sys/firmware/** rwklx,
%   deny /sys/kernel/security/** rwklx,
% }

%%%%%%%%%%%%%%%%%%%%%%%%%%%% CVE-2022-0492 %%%%%%%%%%%%%%%%%%%%%%%%%%%%%%%%%%
% include <tunables/global>
%
% profile docker-default-mount flags=(attach_disconnected,mediate_deleted) {
%
%   #include <abstractions/base>
%
%   ptrace peer=@{profile_name},
%
%   network,
%   capability,
%   file,
%   umount,
%
%   deny @{PROC}/* w,  # deny write for all files directly in /proc (not in a subdir)
%   # deny write to files not in /proc/<number>/** or /proc/sys/**
%   deny @{PROC}/{[^1-9],[^1-9][^0-9],[^1-9s][^0-9y][^0-9s],[^1-9][^0-9][^0-9][^0-9]*}/** w,
%   deny @{PROC}/sys/[^k]** w,  # deny /proc/sys except /proc/sys/k* (effectively /proc/sys/kernel)
%   deny @{PROC}/sys/kernel/{?,??,[^s][^h][^m]**} w,  # deny everything except shm* in /proc/sys/kernel/
%   deny @{PROC}/sysrq-trigger rwklx,
%   deny @{PROC}/kcore rwklx,
%   deny @{PROC}/mem rwklx,
%   deny @{PROC}/kmem rwklx,
%
%   mount,
%
%   deny /sys/[^f]*/** wklx,
%   deny /sys/f[^s]*/** wklx,
%   deny /sys/fs/[^c]*/** wklx,
%   deny /sys/fs/c[^g]*/** wklx,
%   deny /sys/fs/cg[^r]*/** wklx,
%   deny /sys/firmware/** rwklx,
%   deny /sys/kernel/security/** rwklx,
% }

%%%%%%%%%%%%%%%%%%%%%%%%%%%% CVE-2019-5736 %%%%%%%%%%%%%%%%%%%%%%%%%%%%%%%%%%
% include <tunables/global>
%
% profile docker-default-proc flags=(attach_disconnected,mediate_deleted) {
%
%   #include <abstractions/base>
%
%   ptrace,
%
%   network,
%   capability,
%   file,
%   umount,
%
%   deny @{PROC}/* w,  # deny write for all files directly in /proc (not in a subdir)
%   # deny write to files not in /proc/<number>/** or /proc/sys/**
%   # deny @{PROC}/{[^1-9],[^1-9][^0-9],[^1-9s][^0-9y][^0-9s],[^1-9][^0-9][^0-9][^0-9]*}/** w,
%   deny @{PROC}/sys/[^k]** w,  # deny /proc/sys except /proc/sys/k* (effectively /proc/sys/kernel)
%   deny @{PROC}/sys/kernel/{?,??,[^s][^h][^m]**} w,  # deny everything except shm* in /proc/sys/kernel/
%   deny @{PROC}/sysrq-trigger rwklx,
%   deny @{PROC}/kcore rwklx,
%   deny @{PROC}/mem rwklx,
%   deny @{PROC}/kmem rwklx,
%
%   deny mount,
%
%   deny /sys/[^f]*/** wklx,
%   deny /sys/f[^s]*/** wklx,
%   deny /sys/fs/[^c]*/** wklx,
%   deny /sys/fs/c[^g]*/** wklx,
%   deny /sys/fs/cg[^r]*/** wklx,
%   deny /sys/firmware/** rwklx,
%   deny /sys/kernel/security/** rwklx,
% }
