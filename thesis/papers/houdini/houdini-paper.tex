\documentclass[letterpaper,twocolumn,10pt]{article}
\usepackage{../usenix-2020-09}
\usepackage{../huzaifa}
\usepackage{enumitem}
\usepackage{cite}
\usepackage{float}
% Silence annoying warning about nonfrench spacing
\microtypecontext{spacing=nonfrench}

% Enum with custom labels for design Goals
\newlist{dgenum}{enumerate}{1}
\setlist[dgenum]{label=\arabic*., ref=\arabic*}
\crefname{design goal}{Design Goal}{Design Goals}
\crefalias{dgenumi}{design goal}

% 🔹 Add these packages in the preamble
\usepackage{pifont}   % For checkmarks and cross marks
\usepackage{xcolor}   % For colored text

\newcommand{\cmark}{\textcolor{green}{\ding{51}}}  % ✅ Define checkmark
\newcommand{\xmark}{\textcolor{red}{\ding{55}}}    % ❌ Define cross mark

%-------------------------------------------------------------------------------
\begin{document}
%-------------------------------------------------------------------------------

%don't want date printed
%\date{}

% make title bold and 14 pt font (Latex default is non-bold, 16 pt)
\title{\Large \bf Houdini: Security Benchmarking of Container Confinement}
%for single author (just remove % characters)
% TODO: Authors hidden for anonymous submission
\author{
{\rm Huzaifa Patel}\\
Carleton University
\and
{\rm David Barrera}\\
Carleton University
\and
{\rm Anil Somayaji}\\
Carleton University
} % end author

\maketitle

\begin{abstract}
  While container-based workloads are now a standard part of our cloud infrastructure, container security remains a challenging problem. Container confinement is a particularly pressing problem, as without it, a single vulnerable application can be used to compromise entire clusters of containers. While we have many technologies that can be used to secure containers, currently there is no easy way to determine whether a given Docker configuration provides even a basic level of protection. Here we present \houdini, a security benchmark for container confinement. Much as network scanning tools can help catch misconfigured firewalls, \houdini can check whether a given Docker container configuration properly enforces confinement. While it can be used to test deployable containers, \houdini is optimised for testing and comparing container confinement technologies. Here we present the motivation, design, implementation, and initial results on running \houdini on a set of different container configurations. By providing a benchmark framework by which container confinement technologies can be evaluated, we believe \houdini can help foster the development of next-generation container confinement technologies.
\end{abstract}

\section{Introduction}

Container-based workloads are now a standard part of our cloud infrastructure.  Unlike hardware virtual machines, containers are extremely lightweight, incurring essentially no overhead versus just running multiple applications on the same host.  Containers, however, allow workloads to be precisely replicated by including all userspace dependencies of an application in one unified image.  While containers help solve many development and deployment problems, they also create a new problem of container confinement.

Linux cgroups, namespaces and separate mountpoints create the illusion of separation under normal conditions.  Yet there are numerous ways for this separation to fail, and when this happens, attackers can disrupt the operation of other containers or even take control of the host.  In principle, containers can be secured using a variety of technologies including SELinux, AppArmor, and seccomp.  These technologies are complex, however, and the confinement problem is subtle.  Best practices for securing containers [CITES] can help; a guide, however, provides no assurance that the resultant configuration prevents container escapes.  Container confinement is considered so problematic that the cloud industry has now developed multiple technologies for running containers within specially secured virtual machines [CITE gVisor, kata containers].  But would such solutions be necessary if we could check to see whether we were properly confining containers?

Test suites are regularly used to verify that systems meet various requirements.  Performance benchmarks are used to compare the performance of hardware and software.  [LIST COMMON BENCHMARKS LIKE SPEC?]  In software development, tests are used to verify that important functionality has not been broken by code changes.  In many organizations, it is not possible to check in code without it first passing a battery of tests.  Test suites are also used to verify the functionality of production systems, for example with uptime monitors that verify that services are performing their designated functions.  While used less frequently, functional test suites are also used to check for security issues in deployed systems.  Network scanners, in particular, are often used to proactively find accidentally enabled services, firewall issues, and insecure software versions.

Here we present the Houdini, the first test suite for verifying container confinement.  Given a docker container description (including both kernel version and specification of userspace filesystem), Houdini will instantiate the container in a standalone QEMU-based virtual machine and perform multiple tests (tricks, in our parlance) to see whether the configuration is vulnerable to known container escape methods.  Houdini is written in Rust and is easily extensible, providing an extension language for writing tricks.  While it can be used to check general host security, Houdini is specialized to the specific requirements of container confinement.

[I WANT TO LIST CONTRIBUTIONS BUT I'M NOT SURE WHAT TO WRITE]

In this paper we describe Houdini's motivation, design, and implementation.  We also present the results of case studies showing how Houdini can be used to detect misconfigured containers that allow for privilege escalation attacks on the host system.  Our hope with Houdini is that it will help with the deployment of better confined containers and will support the development of new technologies to more reliably confine containers.

The rest of this paper proceeds as follows.  In Section~\ref{sec:confinement}, we explain the container confinement problem and associated technologies.  We describe Houdini's design and implementation in Section~\ref{sec:design}. Section~\ref{sec:testing-confinement} explains the tricks Houdini currently implements and their associated vulnerabilities.  In Section~\ref{sec:evaluation} we present case studies showing how Houdini can detect basic misconfigurations.  Section~\ref{sec:related} describes related work, Section~\ref{sec:discussion} discusses the contributions, limitations, and our plans for future work.  Section~\ref{sec:conclusion} concludes.


%% While containers are widely used, despite their name they don't currently offer strong confinement guarantees. Numerous ways for procesess inside a container to "escape".  While originally containers were designed for administration and software distribution purposes, there is increasing interest in improving container confinement.

%% Variety of solutions:
%% \begin{itemize}
%% \item host-based security mechanisms applied to enforce confinement on processes inside containers (SELinux, AppArmor, system call filters)
%% \item hardware virtualization applied to container confinement (gVisor, kata containers - also known as microVMs, )
%% \item new kernel-level security abstractions, often implemented in eBPF (bpfbox, bpfcontain on archive)
%% \end{itemize}

%% Our question: how effective are these container confinement solutions?

%% Needed: a standard methodology for assessing container confinement
%% \begin{itemize}
%% \item formal approach not feasible, systems are too complex and vulnerabilities are in the details
%% \item empirical approach is possible.  Will never be comprehensive, but that is not the goal.  Instead, we want a baseline set of tests (that can be improved over time) that demonstrates effective container confinement.
%% \end{itemize}

%% Contribution: Houdini, first tool for automated testing of container confinement
%% \begin{itemize}
%% \item series of tests for well-known container vulnerabilities
%% \item designed to facilitate repeatable experiments using qemu-based virtual machines
%% \item easily extensible to allow for new tests to be added, making Houdini more comprehensive over time
%% \end{itemize}
%% (NOTE: we should specify how to "version" Houdini so we can talk about both older and newer versions of Houdini.  So we don't just want a Houdini score, it should be a versioned Houdini score.)

\section{Container Confinement Problem}
\label{sec:confinement}
% why is this a hard problem

% what are the approaches that are used today to confine containers
\begin{itemize}
   \item explain what a container actually is
   \item what you get with namespaces \& cgroups
   \item define containers and the game of escaping containers
\end{itemize}

\section{Related Work}%
\label{sec:related}

At the time of this writing, to our knowledge, \houdini is the first comprehensive approach for evaluating and comparing \textit{container confinement mechanisms}. Much of the container security literature to date has focused on either vulnerability scanning of container images, building offensive/attack tools for container escapes, or proposing best practices for securing container runtimes. While these papers, tools, and documents do not directly share our research objectives, they still broadly fit into the container security landscape. Additionally, research on VM escapes and sandbox escapes has provided valuable insights into isolation vulnerabilities, many of which relate to container security. Techniques used to break out of hypervisors and sandboxed environments often share similarities with container escape methods, which the importance of strong confinement mechanisms. The remainder of this section does not aim to exhaustively enumerate all container security tools and systems, but rather to highlight the various salient methods, including prior research on times the confinement of VMs and sandboxes have been escaped as it is  related to work.

\noindent\textbf{Vulnerability Scanners.} Many free, open, and closed-source tools exist for identifying the presence of known vulnerabilities in container images. Shu \etal~\cite{shu2017study} developed a vulnerability scanning tool to scan Docker Hub container images at scale. They found that on average, official and community images have concerning amounts of vulnerabilities (180+), and that these vulnerabilities remain unpatched for hundreds of days. Their study utilized the Docker Image Vulnerability Analysis (DIVA) framework, which automates the discovery, download, and analysis of over 300,000 Docker images. The findings revealed that more than 80\% of both official and community images contained at least one high-severity vulnerability, and many images were not updated for extended periods. Moreover, vulnerabilities were often inherited from parent images to child images, further increasing security risks.

\noindent\textbf{Best Practices.} Users in search of (often high-level) advice on how to improve the security of their container deployments can consult one of many best practices guides available online. One such guide is the Center for Interent Security (CIS) Docker Benchmark \cite{CIS}. In our experience, guides tend to offer generic advice (e.g., ``\textit{Only allow read access to the root filesystem}''. This is problematic because it oversimplifies container security. Generic advice like "Only allow read access to the root filesystem" may seem straightforward, but it doesn't account for the specific context or unique configuration of an individual deployment. Such one-size-fits-all recommendations can lead users to believe that simply following these broad guidelines is enough to secure their systems, potentially leaving critical vulnerabilities unaddressed. In reality, effective container security requires a nuanced, tailored approach that considers the particular needs and threat models of each environment. Houdini fills the gap between generic advice and practical security by enabling you to objectively evaluate the effectiveness of your container confinement measures.

\noindent\textbf{Offensive Tools.} The offensive security community has developed a range of specialized tools aimed at identifying and exploiting vulnerabilities within containerized environments. Containers, while providing strong isolation, are not immune to attacks, and these tools focus on testing the robustness of container defenses. CDK\footnote{\url{https://github.com/cdk-team/CDK}} and DEEPCE\footnote{\url{https://github.com/stealthcopter/deepce}} are two widely used, open-source container penetration testing toolkits that leverage a collection of known exploits to gain persistence, escape the container, and gather sensitive information about the container environment. These tools aim to bypass isolation mechanisms like namespaces, cgroups, and SELinux, often using a variety of aggressive techniques to identify weaknesses in container security. CDK specifically uses any available method to circumvent security measures, providing a broad toolkit to test the limits of a container’s defense. Meanwhile, DEEPCE, developed by stealthcopter, focuses on privilege escalation and container enumeration, attempting to exploit specific weaknesses to achieve container escapes, thereby compromising the host system. In contrast, Houdini takes a more systematic and structured approach to container security testing. Rather than focusing solely on exploiting vulnerabilities, Houdini allows for the direct comparison of defensive techniques by clearly defining the testing environment, outlining the exploit steps, and documenting the expected outputs.

\noindent\textbf{Virtual Machine Escapes} VM escapes occur when an attacker successfully breaks out of a virtualized confined environment, gaining unauthorized access to the host system. Given that hypervisors manage multiple VMs on shared hardware, such escapes can lead to host compromise, cross-VM attacks, and lateral movement within cloud environments. One of the most notable VM escape vulnerabilities is VENOM (CVE-2015-3456) \cite{abdoul2018novel}, which exploited a flaw in the virtual floppy disk controller (FDC) implementation in QEMU, affecting platforms like Xen, KVM, and VirtualBox. Other significant VM escape vulnerabilities include Cloudburst (2009), which exploited VMware's virtual graphics driver; CVE-2018-6981 and CVE-2018-6982, which targeted VMware ESXi, Workstation, and Fusion \cite{vmware2018escape}; and multiple VirtualBox vulnerabilities (2017) enabling guest-to-host code execution \cite{virtualbox2017bugs}. Additionally, side-channel attacks like MDS/ZombieLoad (2019) demonstrated how speculative execution flaws could be leveraged across VMs \cite{mds2019attack}. More recent research has uncovered weaknesses in encrypted virtualization, such as the SEVered attack (2018), which extracted plaintext memory from AMD SEV-protected VMs \cite{severed2018}, and the Heckler attack (2024), which manipulated non-timer interrupts to compromise confidential VMs in AMD SEV-SNP and Intel TDX \cite{heckler2024}.

\noindent\textbf{Sandbox Escapes} Sandboxing is a critical security mechanism for restricting the execution of untrusted code, commonly used in browsers, mobile apps, and security tools. A sandbox escape occurs when an attacker bypasses these restrictions, executing arbitrary code with higher privileges and potentially leading to full system compromise. Modern web browsers heavily rely on sandboxing to isolate JavaScript execution from the underlying OS. However, several high-profile vulnerabilities have enabled sandbox escapes. For example, CVE-2019-5786 \cite{chromezeroday}, a use-after-free vulnerability in the FileReader API of Google Chrome, allowed attackers to escape Chrome's sandbox and execute code on the host. Other notable sandbox escape vulnerabilities include CVE-2022-26706, which exploited a flaw in macOS's App Sandbox to execute code outside of restricted permissions \cite{macossandboxescape}; CVE-2023-36719, a stack corruption vulnerability in Windows OS that enabled escape from the Chromium sandbox \cite{chromiumsandboxescape}; and CVE-2024-4040, which was actively exploited to break out of CrushFTP’s Virtual File System (VFS) \cite{crushftpsandboxescape}. Additionally, vulnerabilities in Judge0, an online code execution system, exposed systems to full takeover by allowing attackers to escape its sandbox \cite{judge0sandboxescape}. Research has also uncovered critical flaws in vm2, a JavaScript sandbox library, which could be exploited to bypass its security restrictions and gain remote code execution \cite{vm2sandboxescape}.
\section{Linux Confinement Mechanisms}
\label{sec:linux-confinement-mechanisms}

On Linux, process containment relies on several key technologies, including Unix, cgroups, namespaces, Linux capabilities, apparmor, selinux, and seccomp. Below, we provide an overview of each of these mechanisms.

\textbf{cgroups}
Cgroups, or control groups, are a Linux kernel feature that allows system administrators to allocate and manage system resources (such as CPU, memory, disk I/O, and network bandwidth) for processes or groups of processes. When you create a cgroup, you are essentially grouping together processes and imposing resource constraints on them. This is crucial for resource management and isolation in multi-user or containerized environments. For instance, cgroups allow you to set a limit on the amount of memory a set of processes can use, preventing any single process from consuming excessive resources and affecting other processes or the system as a whole. Additionally, cgroups allow for the prioritization of resources, ensuring that critical processes receive more CPU or memory than less important ones. They also offer monitoring capabilities to track how much of each resource a process or cgroup is consuming. This feature is extensively used in containerization technologies like Docker, where containers are assigned their own resource limits via cgroups to ensure fairness and prevent resource hogging.

\textbf{namespaces}
Namespaces are another crucial Linux kernel feature that provides process isolation. A namespace isolates certain aspects of a system for a process or group of processes, ensuring that they have their own independent environment. There are several types of namespaces, including PID, network, mount, user, and UTS namespaces. For example, a PID namespace allows processes within it to have their own process IDs, meaning that processes in different PID namespaces can have the same process ID without interfering with each other. Similarly, network namespaces allow processes to have their own network stack, which includes IP addresses, routing tables, and network interfaces. This isolation is fundamental for containers, as it ensures that processes running in one container cannot see or interact with processes in another, even though they may be running on the same physical host. Namespaces, combined with cgroups, provide the foundation for building lightweight, secure containers in Linux systems.

\textbf{Linux capabilties}
Linux capabilities are a fine-grained security mechanism that splits the traditional superuser (root) privileges into smaller, more manageable pieces. Instead of giving a process full root access, Linux capabilities allow for the assignment of specific privileges to processes, such as the ability to bind to a network port or change the system time, without granting full administrative control. This is particularly useful for increasing system security, as it limits the potential damage caused by vulnerabilities in applications. For example, a web server might require the ability to bind to port 80 but should not need full root privileges. By assigning only the necessary capabilities to a process, the security surface area is reduced, making it harder for an attacker to exploit. This approach is commonly used in containerized environments, where processes are run with just the capabilities they need to function.

\textbf{Seccomp (Secure Computing Mode)}
Seccomp is a security feature in the Linux kernel that allows administrators to filter and restrict the system calls that processes can make. It provides a mechanism to limit the actions that a process can perform, which is particularly useful in reducing the attack surface of applications, especially in containerized environments. With seccomp, you can create a whitelist or blacklist of system calls, effectively controlling which parts of the operating system a process can access. For instance, you could block potentially dangerous system calls like execve, which is used to execute programs, or restrict the ability to open files by denying access to certain file descriptors. This reduces the risk of an attacker exploiting a process to gain control over the system. Seccomp is often used in combination with other security mechanisms like AppArmor and SELinux to provide an additional layer of protection.

\textbf{App Armor (Application Armor)}
AppArmor is a Linux security module that provides Mandatory Access Control (MAC) by enforcing security policies on individual programs. It restricts the actions that applications can perform by defining profiles that specify which files, directories, and capabilities the program can access. AppArmor operates by using a set of predefined or custom profiles, which are attached to programs, to limit their access to system resources. For example, an AppArmor profile for a web server might restrict the server’s access to specific directories and prevent it from executing arbitrary binaries or making network connections to external IP addresses. AppArmor is easier to configure and manage than some other security modules, and it provides a level of protection by ensuring that even if an application is compromised, the attacker has limited access to the system.

\textbf{selinux}
SELinux is a security module that provides more granular control than traditional discretionary access control (DAC) by using mandatory access control (MAC). It enforces policies that define which processes or users can access specific files or resources, based on security contexts. These contexts are applied through SELinux policies, which restrict access even if a process has standard permissions. SELinux is commonly used in high-security environments, such as government or financial systems, to prevent unauthorized access. While powerful, it requires careful configuration to avoid restricting legitimate processes and reduce the risk of privilege escalation.
\section{Design}

Houdini is built using Python 3 and leverages a QEMU VM to provide a controlled testing environment for container security validation. Instead of running containers directly on the host system, Houdini spins up a dedicated VM, ensuring that any security vulnerabilities or container escapes remain isolated and do not affect the host system. This setup allows for highly reproducible security tests while minimizing unintended side effects on the underlying infrastructure.  

At the core of Houdini’s architecture, a Flask-based server runs on the host operating system (not inside the VM). This server acts as a bridge between the host and the QEMU virtual machine, facilitating communication between them. The Flask server is responsible for sending commands and configuration data to the guest VM and receiving trick execution results. When a trick is initiated, the **Houdini client on the host instructs the VM to start a new test. The VM then launches a container with the specified security configurations and executes the trick inside it.  

Once the trick completes, the VM collects the results and sends them back to the host via the Flask server. The host then processes this data, logs the results, and determines whether the container security mechanisms functioned as expected. This structured workflow ensures that Houdini remains independent of the host system’s container runtime while providing a reliable and repeatable environment for testing container confinement mechanisms.

\label{sec:design}

\subsection{Design Goals}

\houdini is designed to provide a systematic and reproducible approach to testing container confinement mechanisms. One of its primary goals is to evaluate whether container security features, such as namespaces, cgroups, seccomp, and Linux capabilities are effectively enforcing isolation. Many security tools rely on theoretical guarantees or static analysis, but Houdini focuses on empirical validation by executing controlled security tests, known as \enquote{tricks}, that verify the correct enforcement of security mechanisms. By doing so, \houdini ensures that container security mechanisms are correctly configured.

One of the goals of \houdini is not only for testing container isolation but also as a means to optimize container configurations. By systematically simulating various tasks within Docker containers, Houdini can help determine the minimal set of privileges required for a container to perform its intended function. For example, administrators can start with a container configured with broad privileges and then incrementally reduce permissions, such as Linux capabilities, seccomp filters, or resource limits, while monitoring task success. Houdini’s testing framework will identify the point at which functionality is maintained while unnecessary privileges are removed, thereby pinpointing the least privilege configuration that still allows the container to operate effectively. This approach is highly beneficial because it adheres to the principle of least privilege. Minimizing the privileges granted to a container reduces its attack surface, making it significantly harder for an attacker to exploit vulnerabilities.

The following are additional goals of Houdini:

\textbf{Verifying Whether Security Mechanisms Are Truly Working as Expected:}
Many security mechanisms in containerized environments, such as Linux namespaces, cgroups, seccomp filters, and capabilities—are intended to enforce process isolation and prevent privilege escalation. However, just because a security feature is enabled does not necessarily mean it is functioning correctly. Misconfigurations, improper runtime enforcement, and subtle implementation flaws can lead to unexpected security gaps. Houdini helps verify whether these security mechanisms are actually enforcing the intended restrictions by executing controlled tests (tricks) that attempt to bypass or manipulate confinement policies. If a test succeeds in breaking out of isolation, it provides direct evidence that the security mechanism is not functioning as expected.


\textbf{Identifying Gaps Between Intended Confinement Policies and Actual Enforcement:}
There is often a discrepancy between security policies that administrators configure and what is actually enforced at runtime. For example, a security policy may specify that a container should not have network access, but due to a misconfigured Docker setting, the container may still be able to establish outbound connections. Similarly, a container may be configured with resource constraints (e.g., CPU and memory limits via cgroups), but in practice, those constraints may be ineffective or bypassable. Houdini systematically tests these security policies against real execution environments to uncover where enforcement fails. By detecting these discrepancies, organizations can adjust their configurations and improve security postures, ensuring that containers remain properly confined under expected conditions.


\textbf{Challenging Misleading Security Assumptions with Empirical Testing:}
Container security is often discussed in terms of theoretical guarantees, with many security best practices based on assumptions rather than direct testing. For instance, documentation may state that a container running in unprivileged mode cannot escalate to root, but Houdini actively tests whether that is actually the case under different container configurations. By running real-world security validation instead of relying solely on security claims, Houdini helps challenge misleading assumptions and exposes where security mechanisms fail in practice. This is particularly important as container escapes and privilege escalation exploits continue to emerge, demonstrating that security controls do not always work as expected. By grounding security claims in measurable results rather than theory, Houdini helps researchers and practitioners develop more robust container security strategies.




In order to achieve the aforementioned goals, we designed \houdini with the following in mind:
\begin{dgenum}
  \item \label{dg:repro}\textit{Reproducible Results.} A key goal of \houdini is to ensure that tests yield consistent and repeatable results. By controlling system variables and maintaining a structured testing environment, Houdini allows researchers and practitioners to compare results across different system configurations and security policies.

  \item \label{dg:container}\textit{Separation from the Host.} \houdini is designed to run tests inside a controlled containerized environment within a virtual machine (VM), ensuring that even if a test triggers a security vulnerability, it does not compromise the host system running the tests. This design choice enhances the safety of security evaluations, preventing unintended side effects on the underlying infrastructure.

  \item \label{dg:test}\textit{Test Case Expressiveness.} \houdini test cases (called \enquote{tricks}) should be maximally expressive, such that some combination of steps can be used to achieve and test any desired result. It should be possible to define a new \houdini trick and modify existing tricks without modifying the \houdini binary. Moreover, it should be clear from reading a defined trick precisely what steps are involved, the consequences of each step passing or failing, and the overall nature of the exploit being tested.

  \item \label{dg:failures}\textit{Focus on Observing Failures, Not Preventing Them.} \houdini is built to identify security weaknesses rather than defend against attacks.
\end{dgenum}

\subsection{Security of the Testing Environment}

If Houdini were compromised during testing, it would not necessarily pose a threat to the host system due to the design of the testing environment. Houdini is intentionally designed to operate within a controlled, containerized environment inside a VM. This means that even if a test were to exploit a vulnerability within the containerized environment, the impact would be contained within the VM, isolated from the host system and other containers. The VM itself serves as an additional layer of security, providing a buffer that prevents the compromise from reaching the host infrastructure.

\subsection{Design of Tricks}

\begin{figure}
  \label{fig:state-machine}
  \includegraphics[width=1\linewidth]{figs/houdini-state-machine.pdf}
  \caption{A state machine diagram of Houdini running a Trick.}
\end{figure}

In containerized environments, security challenges often arise from vulnerabilities within specific components of the architecture. As such, understanding the relationships between the different components—such as container registries, images, runtime, namespaces, cgroups, and various security modules like eBPF, seccomp, and Linux capabilities—is crucial for identifying potential attack surfaces. This is where a comprehensive architecture diagram comes into play.

\begin{figure}
  \label{fig:architecture}
  \includegraphics[width=1\linewidth]{figs/exploit-coverage.pdf}
  \caption{An architectural diagram of a container deployment environment depicting the attack surface created by its various components.}
\end{figure}

Figure 2 outlines the key components of a containerized system, highlighting the critical paths and interactions that security tests should focus on. Security testing should aim to cover these components extensively to ensure that vulnerabilities in any of these layers are adequately addressed. However, some components, like network interfaces, system calls, and cgroups, are more frequently targeted in real-world attacks and thus warrant more attention in testing. The diagram will guide our understanding of which components to prioritize during testing, ensuring that our coverage is both thorough and effective.

\section{Implementation of Houdini Tricks}
\label{sec:implementation}

Exploits that Houdini can test are divided into three separate files: a configuration file, a Python file, and a Dockerfile. Each Houdini trick begins with a configuration file (see Listing 1), which contains the Docker container's settings and environment configuration. The configuration file defines various Docker-specific parameters that are directly fed to the Docker API to start a container with the desired settings. Once the configuration is parsed, Houdini communicates these settings to the Docker API, using Docker's docker run command to initiate the container with these parameters. This ensures that the container is launched with the exact environment specified, allowing for consistent testing of vulnerabilities such as CVE-21616.


\begin{listing}[h]
  \caption{Configuration file for CVE-2024-21616.}
  \label{lst:configuration-file}
  \begin{minted}[frame=lines,framesep=2mm, fontsize=\scriptsize, breaklines=true]{yaml}
  name: CVE-21616
    name: CVE-21616
    dockerfile:
      - path: Dockerfile
    dependencies:
      - server: False
    docker_config: 
      - network_mode: bridge
      - read_only: False
      - security_opt: ["no-new-privileges"]
      - pid_mode: null
      - cpu_shares: null
      - volumes : {"/proc": {"bind": "/host_proc", "mode": "ro"}}
      - mem_limit: null
      - cpuset_cpus: null
      - cpu_quota: null
      - cpu_period: null
      - cap_add: []
      - cap_drop: []
      - privileged: False
      - user: root
      - pids_limit: null
    trick:
      - path: /houdini/tricks/HostMount
  \end{minted}
\end{listing}



The second component is a Python file that contains the actual exploit for the trick. This Python file is executed inside the container and is responsible for carrying out the trick scenario, such as testing privilege escalation or container breakout attempts.


\begin{listing}[h]
  \caption{}
\label{lst:pythonfile}
  \begin{minted}[frame=lines,framesep=2mm, fontsize=\scriptsize, breaklines=true]{yaml}
    import os

    # Define the relative path
    relative_path = '../../../../../'

    # Change the current working directory
    try:
        os.chdir(relative_path)
        print(f"Successfully changed directory to: {os.getcwd()}")
    except FileNotFoundError as e:
        print(f"Error: {e}")
    except PermissionError as e:
        print(f"Error: {e}")
    except Exception as e:
        print(f"Unexpected error: {e}")
  \end{minted}
\end{listing}


Finally, the Dockerfile is used to configure and set up the container's environment. It specifies how the container should be built, including copying necessary files (like the Python file) into the container and installing required dependencies. The Dockerfile also configures the container with the settings defined in the configuration file, ensuring the container is set up correctly to run the exploit.

\begin{listing}[H]
  \caption{}
\label{lst:dockerfile}
  \begin{minted}[frame=lines,framesep=2mm, fontsize=\scriptsize, breaklines=true]{yaml}
    FROM ubuntu:20.04
    RUN apt-get update -y
    WORKDIR /proc/self/fd/9
    CMD ["bash", "-c", "ls ../../../../../houdini/tricks/HostMount"]
  \end{minted}
\end{listing}

This modular design—comprising the configuration file, Python script, and Dockerfile—provides flexibility to define and test various container escapes and vulnerabilities. It also allows users to easily customize or extend existing tricks by modifying any of the three components to suit their needs.
% \section{Testing Confinement}
\label{sec:testing-confinement}

\begin{table*}
  \caption{\todo{table caption}}
  \label{tab:classification}
  \centering \small
  \begin{tabular}{lllll}
    \toprule
    \bfseries No\@. & \bfseries Exploit & & \bfseries CVE No\@. & \bfseries Attack Vectors \normalfont{(See \Cref{fig:architecture})} \\
    \midrule
    1 & Mounted Docker Socket               & CITE & ---            & C, D, E, N \\
    2 & Mounted \texttt{/etc/passwd}        & CITE & ---            & C, D, G, K \\
    3 & Pipes Read-Only Overwrite           & CITE & CVE-2022-0847  & O, S \\
    4 & Cgroup Release Agent Code Execution & CITE & CVE-2022-0492  & M, O, R \\
    5 & Apache Path Traversal               & CITE & CVE-2021-42013 & D, I \\
    6 & \texttt{runc} Binary Overwrite      & CITE & CVE-2019-5736  & A, B, C, D, H, K \\
    \bottomrule
  \end{tabular}
\end{table*}

\todo{KEVIN: Explain some technical details for each exploit. Refer back to \Cref{tab:classification} as necessary.}

% https://unit42.paloaltonetworks.com/breaking-docker-via-runc-explaining-cve-2019-5736/
% https://nvd.nist.gov/vuln/detail/CVE-2019-5736
% exploits the container runtime runc [C, E?], the procfs filesystem [H, K], process IDs [P]
% RCE, elevation of privilege, DoS
\noindent\textbf{CVE-2019-5736.} This CVE\cite{addtheref} is a remote code execution vulnerability in the \texttt{runc} binary. The PoC exploits the way \texttt{runc} creates a process in the container called \texttt{runcInit} to run a specified command. The \texttt{procfs} file system contains a symlink to the binary being executed in \texttt{/proc/self/exe} which points to the \texttt{runc} executable on the host. A malicious container overwrites its own \texttt{/bin/sh} binary to the \texttt{/proc/self/exe} interpreter \texttt{\#!/proc/self/exe}. Once a user executes the overwritten \texttt{/bin/sh} binary, the interpreter calls the \texttt{runc} binary whose pid is then captured. Malicious code will then obtain the file descriptor from \texttt{/proc/runc-pid/exe} and overwrite the contents of the \texttt{runc} binary on the host.

% https://blog.qualys.com/vulnerabilities-threat-research/2021/10/27/apache-http-server-path-traversal-remote-code-execution-cve-2021-41773-cve-2021-42013
% https://nvd.nist.gov/vuln/detail/CVE-2021-42013
% exploits a specific vulnerable application in the container [D] through HTTP urls containing file traversal character sequences
% RCE
\noindent\textbf{CVE-2021-42013.} This is a remote code execution vulnerability in the Apache HTTP server container (versions 2.4.49 and 2.4.50). The exploit is a path traversal vulnerability that takes advantage of the server being unable to detect the path traversal characters ``../'' in a URL. This may occur when the second dot is replaced by its unicode representation ``\%2e''. By requesting a URL with several path traversal sequences and a binary to execute, contents of files can be retrieved from the container's local filesystem.

% https://dirtypipe.cm4all.com/
% https://nvd.nist.gov/vuln/detail/CVE-2022-0492
% exploits the cgroup file system [K, R?], cgroups [M], processes [I], special files that interact with the cgroups [H?], the way container filesystems are stored on the host [O, R?]
% RCE, elevation of privilege, information disclosure
\noindent\textbf{CVE-2022-0492.} A bug in the kernel's \detokenize{cgroup_release_agent_write cgroupv1} filesystem code can be exploited to enable a container escape. A file called \detokenize{release_agent} gets executed when a process in the cgroup terminates if \detokenize{notify_on_release} is enabled. The exploit involves mounting the cgroups filesystem on the host to the container and creating a directory which creates a new cgroup. Creating the \detokenize{notify_on_release} file in the new directory and writing a 1 to it tells the host to run the \detokenize{release_agent} file when a process in the custom cgroup terminates. Modifying the \detokenize{release_agent} file with a script that executes a malicious binary on the container from the host's filesystem allows the malicious container to successfully escape upon termination of a process in the custom cgroup.

% https://sysdig.com/blog/detecting-mitigating-cve-2022-0492-sysdig/
% https://nvd.nist.gov/vuln/detail/CVE-2022-0847
% exploits a vulnerability in the pipe.c code [S] that allows data within a spliced page cache to be overwritten if pipe is set up in a specific way
% RCE, elevation of privilege, DoS
\noindent\textbf{CVE-2022-0847.} This is an escalation of privilege Linux kernel vulnerability caused by a bug in the kernel's pipe.c code introduced in version 5.8. The exploit utilizes the pipe buffers flag \detokenize{PIPE_BUF_FLAG_CAN_MERGE} which allows for data to be overwritten in a page cache from a spliced file if the pipe buffer is prepared in a specific way. The flag gets set by filling the buffer with random data. The pipe buffer can then be emptied and then a file can be spliced into the pipe. Writing data into the pipe buffer now overwrites the data in the file. In order to escape a container with this exploit, the setup is similar to that of CVE-2019-5736 above where a malicious container waits for runc to execute in the container. The file descriptor is grabbed from the procfs and overwritten using the dirty pipe vulnerability.

We selected the exploits above based on several criteria. (1) The exploits must have a proof of concept readily available online; (2) the exploits must cover various components of the container deployment environment (see \Cref{fig:architecture}); and (3) the exploits must be high impact and severity defined by the CVS standards. The exploits chosen all fit within these criteria and can be tested repeatedly with different system configurations using \houdini.

The exploits are run in privileged and unprivileged mode with one of seccomp or apparmor securing the container. Most exploits are blocked successfully by using these security systems, but these security systems might not be configured or enabled by default. For example, kernels can come with apparmor on the system but it is disabled by default on boot and the service must be re-enabled by manipulating GRUB boot parameters. Using Docker info on a system lets a user know what container security mechanisms are available to the user. If apparmor is not installed on the system or the service is not enabled on boot through GRUB, then seccomp is usually the only security mechanism that can be deployed.

The default policy files for both apparmor and seccomp are used during testing. The default policy for apparmor most importantly restricts write permissions in the sensitive \texttt{/proc} and \texttt{/sys} file sysytems, as well as denying mounting operations. The default seccomp profile allows for all but 44 of the system calls to pass through and can allow more based on the container's capabilities. The system calls that are filtered out and restricted are focused around sensitive operations that can interact directly with the host kernel and potentially modify its behaviour (kernel modules, system time, reboot, etc.).

Tests were run on Ubuntu 22.04 with kernel version 5.15.0-50-generic. The docker engine is version 18.09.1 and runc version is 1.1.0+dev. For virtualization we used QEMU version 6.2.0 and buildroot version 2022.02.

We manually created a trick yaml file for each exploit outlining the environment and the commands for the host and container to perform. If required for the exploit, a new environment is created using the buildroot to generate a new virtual machine with a configurable linux kernel and filesystem. The versions for the kernel and specific packages that are required to run the exploit successfully, such as docker and runc, that are specified in the exploit yaml are passed to buildroot as parameters. Houdini is also included as a package and launches when the virtual system boots. QEMU runs the virtual environment and creates a virtual socket (vsock) connection between the host and the VM for status reporting and management. Once \houdini on the VM boots, a connection is made to \houdini running on the host. The trick is then sent via the vsock connection and parsed by the \houdini client on the VM and for the remainder of the trick, the VM is acting as the host. Once the trick finishes execution, the results are sent back to the host \houdini using the vsock connection and the VM is shut down.

The exploit CVE-2019-5736 is set up with a malicious container that runs a script on start to wait for the /bin/sh binary to execute and then in turn execute its own binary to overwrite the host runc. The vulnerable runc version being tested is version 1.0.0-rc6. On the host side, the user spins up the container and attempts to exec into the container using "docker exec -it ID sh". The exploit is considered a success if the runc binary on the host system has been compromised.

CVE-2021-42013 has a different threat model where the container image is vulnerable but not necessarily malicious. The exploit begins with a vulnerable apache httpd image being launched and then on the host side, the exploit binary is called using the container's IP address and a command as input. The exploit is considered a success if the command runs and data is received containing the response from the container.

Unlike the previous two exploits, CVE-2022-0492 does not utilize any malicious binaries. The exploit takes place with commands executed entirely within the container. A privileged container is launched and the cgroup rdma folder from the host cgroup filesystem is mounted in the container allowing for manipulation of the cgroup's release agent file. The exploit is considered successful if all commands execute successfully with no errors.

%%%%%%%%%%%%%%%%%%%% CVE-2022-0492 %%%%%%%%%%%%%%%%%%%%%%
% mkdir /tmp/mountest
% mount -t cgroup -o rdma cgroup /tmp/mountest
% mkdir /tmp/mountest/x
% echo 1 > /tmp/mountest/x/notify_on_release
% host_path=`sed -n 's/.*\perdir=\([^,]*\).*/\1/p' /etc/mtab`
% echo "$host_path/cmd" > /tmp/mountest/release_agent
% echo '#!/bin/sh' > /cmd
% echo "cat /etc/passwd > $host_path/output" >> /cmd
% chmod a+x /cmd
% sh -c "echo \$\$ > /tmp/mountest/x/cgroup.procs"
% cat /output

% display and explain results and how each test was ran

\begin{table*}
 \caption{\todo{table caption, replace success/failure text with glyphs. re-order, grouping unprivileged rows and privileged rows. }}
 \label{tab:results}
 \centering \small
 \begin{tabular}{lllll}
   \toprule
   \bfseries Security & \bfseries CVE-2019-5736 & \bfseries CVE-2021-42013 & \bfseries CVE-2022-0492 & \bfseries CVE-2022-0847 \normalfont{(See \Cref{fig:architecture})} \\
   \midrule
   unprivileged & success & success & failure & success \\
   privileged  & success & success & success & success \\
   unprivileged + apparmor & failure & success & failure & failure \\
   privileged + apparmor & failure & success & failure & failure\\
   unprivileged + seccomp & failure & success & failure & success\\
   privileged + seccomp & success & success &  success & success \\
   \bottomrule
 \end{tabular}
\end{table*}

\noindent\textbf{Why use Houdini.}

Container hardening techniques through confinement mechanisms require both knowledge of the application code and security best practices. Ensuring that their container environment is not vulnerable to known common exploits is a step that many developers will either save for the last part of their development cycle or forget about all together. Testing frameworks allow developers to save time by automatically running a suite of some tests in order to give the developer more insight into how their application behaves. Existing exploit testing frameworks are not designed with the goal of exposing problematic areas in a container's confinement policy or environment. \houdini allows the user to run a series of known vulnerabilities against their container environment to expose issues for the developer to quickly asses and act upon. The security mechanisms that are used to lockdown the container can easily be tested repeatedly throughout the development lifecycle without much overhead for the developer. \houdini also allows for the testing of new confinement mechanisms by academics and researchers with very little required customization to the trick files themselves. \houdini can test any part of the system for vulnerabilities including the kernel, runtime, and the container configuration which all contribute towards making the container vulnerable. This makes \houdini a powerful tool when assisting in the development of container confinement policies and environment testing.

\noindent\textbf{What makes running the exploit possible in apparmor.}

Apparmor and seccomp are integrated with docker to allow a user to further confine their resources and capabilities. When apparmor is enabled on a host kernel, docker will apply a default apparmor policy to a container. The policy name is docker default and it restricts: mounting, most cases of writing to /proc, and most cases of writing to /sys. This rather limited profile does a relatively good job of blocking common container escapes and privilege escalations. Issues mainly arise when a user tries to modify the existing policy, or develop their own without proper consideration. An apparmor profile is developed by specifying resources to allow or deny access to on each line. The profile must then be ran through the apparmor parser which checks for the proper syntax to be followed. Once the parser successfully reads and loads the profile, docker can use it as a value passed to their \detokenize{"--security-opts apparmor=PROFILE_NAME"} flag.

The apparmor parser will not check for whether a specified system path exists or not and if contradictory statements exist, it uses the statement that is more restrictive (i.e. if "mount" and "deny mount" exist in the same profile, it will always use "deny mount" no matter the declared order).

In order for CVE-2022-0492 to successfully operate with the apparmor security enabled on a privileged container, the only line that needs to be modified in the default profile is "deny mount". If a user were to change the "deny mount" to "mount" for either their own usage or as a simple mistake, then the exploit would be able to run successfully.

CVE-2019-5736 requires a little more modification to the docker default profile. Write access to \texttt{/proc/PID} must be allowed and full use of ptrace must be allowed. The exploit can successfully overwrite its own shell binary without any modifications to the default profile. If only write access to \texttt{/proc/PID} is given, the exploit can grab a hold of the runc process ID, but access to ptrace is required as it is used for process control by the system when modifying the \texttt{/proc/PID/exe} file. These two changes to the default apparmor profile allow CVE-2022-0492 to successfully overwrite the runc binary even without the privileged flag.

Using older versions of the container runtime RunC before version 1.0.0-rc8 allows for a different bypass of the apparmor confinement policy. Since apparmor is based off of path names, a malicious user can get around the default apparmor's policy that is locking down a specific path name by simply remounting it to a different path. An example of this is the procfs filesystem mounted on \texttt{/proc}. The Apparmor default profile locks down the procfs filesystem at the defined \texttt{/proc} location. By remounting the procfs filesystem to a different location, this bypasses the pathname based confinement that apparmor offers.

\noindent\textbf{What do the results tell us.}

\todo{docker default security profiles can be overly permissive. Why?}
\todo{seccomp by default allows almost 90\% of systemcalls}
\todo{apparmor allows all network capability}
\todo{default profiles are needed to be generic and more likely overly permissive}
\todo{generally do a good job catching existing exploits by locking down key parts}
\todo{Compromise between security and usability}
\todo{developing profiles can be difficult even with tools such as aa-easyprof}
\todo{testing confinement is made easy}


%%%%%%%%%%%%%%%%%%%%%%%%%%%% DEFAULT %%%%%%%%%%%%%%%%%%%%%%%%%%%%%%%%%%
% include <tunables/global>
%
% profile docker-default flags=(attach_disconnected,mediate_deleted) {
%
%   #include <abstractions/base>
%
%   ptrace peer=@{profile_name},
%
%   network,
%   capability,
%   file,
%   umount,
%
%   deny @{PROC}/* w,  # deny write for all files directly in /proc (not in a subdir)
%   # deny write to files not in /proc/<number>/** or /proc/sys/**
%   deny @{PROC}/{[^1-9],[^1-9][^0-9],[^1-9s][^0-9y][^0-9s],[^1-9][^0-9][^0-9][^0-9]*}/** w,
%   deny @{PROC}/sys/[^k]** w,  # deny /proc/sys except /proc/sys/k* (effectively /proc/sys/kernel)
%   deny @{PROC}/sys/kernel/{?,??,[^s][^h][^m]**} w,  # deny everything except shm* in /proc/sys/kernel/
%   deny @{PROC}/sysrq-trigger rwklx,
%   deny @{PROC}/kcore rwklx,
%   deny @{PROC}/mem rwklx,
%   deny @{PROC}/kmem rwklx,
%
%   deny mount,
%
%   deny /sys/[^f]*/** wklx,
%   deny /sys/f[^s]*/** wklx,
%   deny /sys/fs/[^c]*/** wklx,
%   deny /sys/fs/c[^g]*/** wklx,
%   deny /sys/fs/cg[^r]*/** wklx,
%   deny /sys/firmware/** rwklx,
%   deny /sys/kernel/security/** rwklx,
% }

%%%%%%%%%%%%%%%%%%%%%%%%%%%% CVE-2022-0492 %%%%%%%%%%%%%%%%%%%%%%%%%%%%%%%%%%
% include <tunables/global>
%
% profile docker-default-mount flags=(attach_disconnected,mediate_deleted) {
%
%   #include <abstractions/base>
%
%   ptrace peer=@{profile_name},
%
%   network,
%   capability,
%   file,
%   umount,
%
%   deny @{PROC}/* w,  # deny write for all files directly in /proc (not in a subdir)
%   # deny write to files not in /proc/<number>/** or /proc/sys/**
%   deny @{PROC}/{[^1-9],[^1-9][^0-9],[^1-9s][^0-9y][^0-9s],[^1-9][^0-9][^0-9][^0-9]*}/** w,
%   deny @{PROC}/sys/[^k]** w,  # deny /proc/sys except /proc/sys/k* (effectively /proc/sys/kernel)
%   deny @{PROC}/sys/kernel/{?,??,[^s][^h][^m]**} w,  # deny everything except shm* in /proc/sys/kernel/
%   deny @{PROC}/sysrq-trigger rwklx,
%   deny @{PROC}/kcore rwklx,
%   deny @{PROC}/mem rwklx,
%   deny @{PROC}/kmem rwklx,
%
%   mount,
%
%   deny /sys/[^f]*/** wklx,
%   deny /sys/f[^s]*/** wklx,
%   deny /sys/fs/[^c]*/** wklx,
%   deny /sys/fs/c[^g]*/** wklx,
%   deny /sys/fs/cg[^r]*/** wklx,
%   deny /sys/firmware/** rwklx,
%   deny /sys/kernel/security/** rwklx,
% }

%%%%%%%%%%%%%%%%%%%%%%%%%%%% CVE-2019-5736 %%%%%%%%%%%%%%%%%%%%%%%%%%%%%%%%%%
% include <tunables/global>
%
% profile docker-default-proc flags=(attach_disconnected,mediate_deleted) {
%
%   #include <abstractions/base>
%
%   ptrace,
%
%   network,
%   capability,
%   file,
%   umount,
%
%   deny @{PROC}/* w,  # deny write for all files directly in /proc (not in a subdir)
%   # deny write to files not in /proc/<number>/** or /proc/sys/**
%   # deny @{PROC}/{[^1-9],[^1-9][^0-9],[^1-9s][^0-9y][^0-9s],[^1-9][^0-9][^0-9][^0-9]*}/** w,
%   deny @{PROC}/sys/[^k]** w,  # deny /proc/sys except /proc/sys/k* (effectively /proc/sys/kernel)
%   deny @{PROC}/sys/kernel/{?,??,[^s][^h][^m]**} w,  # deny everything except shm* in /proc/sys/kernel/
%   deny @{PROC}/sysrq-trigger rwklx,
%   deny @{PROC}/kcore rwklx,
%   deny @{PROC}/mem rwklx,
%   deny @{PROC}/kmem rwklx,
%
%   deny mount,
%
%   deny /sys/[^f]*/** wklx,
%   deny /sys/f[^s]*/** wklx,
%   deny /sys/fs/[^c]*/** wklx,
%   deny /sys/fs/c[^g]*/** wklx,
%   deny /sys/fs/cg[^r]*/** wklx,
%   deny /sys/firmware/** rwklx,
%   deny /sys/kernel/security/** rwklx,
% }

\section{Evaluation}
\label{sec:evaluation}

Houdini on:
\begin{itemize}
\item default docker container unpriv
\item default docker container priv
\item same as above, but with AppArmor
\item find "how to secure a container" policies?
\item make sure to pick apart standard docker policies, try to see which pieces contribute what protection
\end{itemize}

\section{Summary of Usecases}
\label{sec:usecases}

\noindent\textbf{Bridging the Gap Between Theoretical and Practical Container Security.}

Houdini provides an objective and practical way to measure container security by testing the actual security properties being enforced, rather than relying on assumptions or theoretical guarantees about what should be enforced.

Many security mechanisms, such as namespaces, cgroups, seccomp, and capabilities are designed to isolate and restrict containers. However, their effectiveness depends on proper implementation and configuration. In many cases, security policies may appear to be correctly applied but fail under real-world attack scenarios.

Houdini ensures that security claims are backed by empirical testing rather than assumptions. It allows researchers and practitioners to:

\begin{dgenum}
\item Verify whether security mechanisms are truly working as expected.
\item Identify gaps between intended confinement policies and actual enforcement.
\item Challenge misleading security assumptions, ensuring that claims about container security are based on real, measurable behavior rather than theoretical models.
\end{dgenum}
\section{Conclusion}
\label{sec:conclusion}

In this paper, we presented Houdini, a security benchmarking framework designed to evaluate the effectiveness of container confinement mechanisms. As container-based workloads continue to dominate cloud infrastructures, ensuring that containers are properly isolated is paramount for maintaining security. Houdini offers a systematic and empirical approach to test and assess various container security configurations, highlighting potential vulnerabilities and misconfigurations that may compromise container isolation. Our evaluation demonstrated the ability of Houdini to detect critical misconfigurations, including those leading to privilege escalation and container escapes. By providing a reliable and reproducible method for testing container confinement, Houdini helps bridge the gap between theoretical security assumptions and practical, real-world behavior.

As container technologies evolve, it is essential to continuously assess and improve their security mechanisms. Houdini's modular design ensures that it remains adaptable to new security challenges and configurations, allowing for ongoing validation and refinement of container security practices. By offering an open framework for container confinement benchmarking, Houdini aims to support the development of more secure container technologies, ultimately contributing to a safer cloud infrastructure for the future.

\bibliographystyle{plain}
\bibliography{refs}

\end{document}
