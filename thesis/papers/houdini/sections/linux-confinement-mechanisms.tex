\section{Linux Confinement Mechanisms}
\label{sec:linux-confinement-mechanisms}

On Linux, process containment relies on several key technologies, including Unix, cgroups, namespaces, Linux capabilities, apparmor, selinux, and seccomp. Below, we provide an overview of each of these mechanisms.

\textbf{cgroups}
Cgroups, or control groups, are a Linux kernel feature that allows system administrators to allocate and manage system resources (such as CPU, memory, disk I/O, and network bandwidth) for processes or groups of processes. When you create a cgroup, you are essentially grouping together processes and imposing resource constraints on them. This is crucial for resource management and isolation in multi-user or containerized environments. For instance, cgroups allow you to set a limit on the amount of memory a set of processes can use, preventing any single process from consuming excessive resources and affecting other processes or the system as a whole. Additionally, cgroups allow for the prioritization of resources, ensuring that critical processes receive more CPU or memory than less important ones. They also offer monitoring capabilities to track how much of each resource a process or cgroup is consuming. This feature is extensively used in containerization technologies like Docker, where containers are assigned their own resource limits via cgroups to ensure fairness and prevent resource hogging.

\textbf{namespaces}
Namespaces are another crucial Linux kernel feature that provides process isolation. A namespace isolates certain aspects of a system for a process or group of processes, ensuring that they have their own independent environment. There are several types of namespaces, including PID, network, mount, user, and UTS namespaces. For example, a PID namespace allows processes within it to have their own process IDs, meaning that processes in different PID namespaces can have the same process ID without interfering with each other. Similarly, network namespaces allow processes to have their own network stack, which includes IP addresses, routing tables, and network interfaces. This isolation is fundamental for containers, as it ensures that processes running in one container cannot see or interact with processes in another, even though they may be running on the same physical host. Namespaces, combined with cgroups, provide the foundation for building lightweight, secure containers in Linux systems.

\textbf{Linux capabilties}
Linux capabilities are a fine-grained security mechanism that splits the traditional superuser (root) privileges into smaller, more manageable pieces. Instead of giving a process full root access, Linux capabilities allow for the assignment of specific privileges to processes, such as the ability to bind to a network port or change the system time, without granting full administrative control. This is particularly useful for increasing system security, as it limits the potential damage caused by vulnerabilities in applications. For example, a web server might require the ability to bind to port 80 but should not need full root privileges. By assigning only the necessary capabilities to a process, the security surface area is reduced, making it harder for an attacker to exploit. This approach is commonly used in containerized environments, where processes are run with just the capabilities they need to function.

\textbf{Seccomp (Secure Computing Mode)}
Seccomp is a security feature in the Linux kernel that allows administrators to filter and restrict the system calls that processes can make. It provides a mechanism to limit the actions that a process can perform, which is particularly useful in reducing the attack surface of applications, especially in containerized environments. With seccomp, you can create a whitelist or blacklist of system calls, effectively controlling which parts of the operating system a process can access. For instance, you could block potentially dangerous system calls like execve, which is used to execute programs, or restrict the ability to open files by denying access to certain file descriptors. This reduces the risk of an attacker exploiting a process to gain control over the system. Seccomp is often used in combination with other security mechanisms like AppArmor and SELinux to provide an additional layer of protection.

\textbf{App Armor (Application Armor)}
AppArmor is a Linux security module that provides Mandatory Access Control (MAC) by enforcing security policies on individual programs. It restricts the actions that applications can perform by defining profiles that specify which files, directories, and capabilities the program can access. AppArmor operates by using a set of predefined or custom profiles, which are attached to programs, to limit their access to system resources. For example, an AppArmor profile for a web server might restrict the server’s access to specific directories and prevent it from executing arbitrary binaries or making network connections to external IP addresses. AppArmor is easier to configure and manage than some other security modules, and it provides a level of protection by ensuring that even if an application is compromised, the attacker has limited access to the system.

\textbf{selinux}
SELinux is a security module that provides more granular control than traditional discretionary access control (DAC) by using mandatory access control (MAC). It enforces policies that define which processes or users can access specific files or resources, based on security contexts. These contexts are applied through SELinux policies, which restrict access even if a process has standard permissions. SELinux is commonly used in high-security environments, such as government or financial systems, to prevent unauthorized access. While powerful, it requires careful configuration to avoid restricting legitimate processes and reduce the risk of privilege escalation.