\section{Summary of Usecases}
\label{sec:usecases}

\noindent\textbf{Bridging the Gap Between Theoretical and Practical Container Security.}

Houdini provides an objective and practical way to measure container security by testing the actual security properties being enforced, rather than relying on assumptions or theoretical guarantees about what should be enforced.

Many security mechanisms, such as namespaces, cgroups, seccomp, and capabilities are designed to isolate and restrict containers. However, their effectiveness depends on proper implementation and configuration. In many cases, security policies may appear to be correctly applied but fail under real-world attack scenarios.

Houdini ensures that security claims are backed by empirical testing rather than assumptions. It allows researchers and practitioners to:

\begin{dgenum}
\item Verify whether security mechanisms are truly working as expected.
\item Identify gaps between intended confinement policies and actual enforcement.
\item Challenge misleading security assumptions, ensuring that claims about container security are based on real, measurable behavior rather than theoretical models.
\end{dgenum}