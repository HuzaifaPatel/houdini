\section{Conclusion}
\label{sec:conclusion}

In this paper, we presented Houdini, a security benchmarking framework designed to evaluate the effectiveness of container confinement mechanisms. As container-based workloads continue to dominate cloud infrastructures, ensuring that containers are properly isolated is paramount for maintaining security. Houdini offers a systematic and empirical approach to test and assess various container security configurations, highlighting potential vulnerabilities and misconfigurations that may compromise container isolation. Our evaluation demonstrated the ability of Houdini to detect critical misconfigurations, including those leading to privilege escalation and container escapes. By providing a reliable and reproducible method for testing container confinement, Houdini helps bridge the gap between theoretical security assumptions and practical, real-world behavior.

As container technologies evolve, it is essential to continuously assess and improve their security mechanisms. Houdini's modular design ensures that it remains adaptable to new security challenges and configurations, allowing for ongoing validation and refinement of container security practices. By offering an open framework for container confinement benchmarking, Houdini aims to support the development of more secure container technologies, ultimately contributing to a safer cloud infrastructure for the future.