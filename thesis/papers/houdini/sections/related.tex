\section{Related Work}%
\label{sec:related}

At the time of this writing, to our knowledge, \houdini is the first comprehensive approach for evaluating and comparing \textit{container confinement mechanisms}. Much of the container security literature to date has focused on either vulnerability scanning of container images, building offensive/attack tools for container escapes, or proposing best practices for securing container runtimes. While these papers, tools, and documents do not directly share our research objectives, they still broadly fit into the container security landscape. The remainder of this section does not aim to exhaustively enumerate all container security tools and systems, but rather to highlight the various salient methods.

\noindent\textbf{Vulnerability Scanners.} Many free, open, and closed-source tools exist for identifying the presence of known vulnerabilities in container images. Shu \etal~\cite{shu2017study} developed a vulnerability scanning tool to scan Docker Hub container images at scale. They found that on average, official and community images have concerning amounts of vulnerabilities (180+), and that these vulnerabilities remain unpatched for hundreds of days. Their study utilized the Docker Image Vulnerability Analysis (DIVA) framework, which automates the discovery, download, and analysis of over 300,000 Docker images. The findings revealed that more than 80\% of both official and community images contained at least one high-severity vulnerability, and many images were not updated for extended periods. Moreover, vulnerabilities were often inherited from parent images to child images, further increasing security risks.

\noindent\textbf{Best Practices.} Users in search of (often high-level) advice on how to improve the security of their container deployments can consult one of many best practices guides available online. One such guide is the Center for Interent Security (CIS) Docker Benchmark \cite{CIS}. In our experience, guides tend to offer generic advice (e.g., ``\textit{Only allow read access to the root filesystem}''. This is problematic because it oversimplifies container security. Generic advice like "Only allow read access to the root filesystem" may seem straightforward, but it doesn't account for the specific context or unique configuration of an individual deployment. Such one-size-fits-all recommendations can lead users to believe that simply following these broad guidelines is enough to secure their systems, potentially leaving critical vulnerabilities unaddressed. In reality, effective container security requires a nuanced, tailored approach that considers the particular needs and threat models of each environment. Houdini fills the gap between generic advice and practical security by enabling you to objectively evaluate the effectiveness of your container confinement measures. In the evaluation section, we leverage the best practices guide \cite{CIS} with \houdini, to rigorously assess whether container isolation mechanisms achieves confinement based on different components that make up the Docker environment.


\noindent\textbf{Offensive Tools.} The offensive security community has developed a range of specialized tools aimed at identifying and exploiting vulnerabilities within containerized environments. Containers, while providing strong isolation, are not immune to attacks, and these tools focus on testing the robustness of container defenses. CDK\footnote{\url{https://github.com/cdk-team/CDK}} and DEEPCE\footnote{\url{https://github.com/stealthcopter/deepce}} are two widely used, open-source container penetration testing toolkits that leverage a collection of known exploits to gain persistence, escape the container, and gather sensitive information about the container environment. These tools aim to bypass isolation mechanisms like namespaces, cgroups, and SELinux, often using a variety of aggressive techniques to identify weaknesses in container security. CDK specifically uses any available method to circumvent security measures, providing a broad toolkit to test the limits of a container’s defense. Meanwhile, DEEPCE, developed by stealthcopter, focuses on privilege escalation and container enumeration, attempting to exploit specific weaknesses to achieve container escapes, thereby compromising the host system. In contrast, Houdini takes a more systematic and structured approach to container security testing. Rather than focusing solely on exploiting vulnerabilities, Houdini allows for the direct comparison of defensive techniques by clearly defining the testing environment, outlining the exploit steps, and documenting the expected outputs.