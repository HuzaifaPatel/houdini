\section{Related Work}%
\label{sec:related}

At the time of this writing, to our knowledge, \houdini is the first comprehensive approach for evaluating and comparing \textit{container confinement mechanisms}. Much of the container security literature to date has focused on either vulnerability scanning of container images, building offensive/attack tools for container escapes, or proposing best practices for securing container runtimes. While these papers, tools, and documents do not directly share our research objectives, they still broadly fit into the container security landscape. Additionally, research on VM escapes and sandbox escapes has provided valuable insights into isolation vulnerabilities, many of which relate to container security. Techniques used to break out of hypervisors and sandboxed environments often share similarities with container escape methods, which the importance of strong confinement mechanisms. The remainder of this section does not aim to exhaustively enumerate all container security tools and systems, but rather to highlight the various salient methods, including prior research on times the confinement of VMs and sandboxes have been escaped as it is  related to work.

\noindent\textbf{Vulnerability Scanners.} Many free, open, and closed-source tools exist for identifying the presence of known vulnerabilities in container images. Shu \etal~\cite{shu2017study} developed a vulnerability scanning tool to scan Docker Hub container images at scale. They found that on average, official and community images have concerning amounts of vulnerabilities (180+), and that these vulnerabilities remain unpatched for hundreds of days. Their study utilized the Docker Image Vulnerability Analysis (DIVA) framework, which automates the discovery, download, and analysis of over 300,000 Docker images. The findings revealed that more than 80\% of both official and community images contained at least one high-severity vulnerability, and many images were not updated for extended periods. Moreover, vulnerabilities were often inherited from parent images to child images, further increasing security risks.

\noindent\textbf{Best Practices.} Users in search of (often high-level) advice on how to improve the security of their container deployments can consult one of many best practices guides available online. One such guide is the Center for Interent Security (CIS) Docker Benchmark \cite{CIS}. In our experience, guides tend to offer generic advice (e.g., ``\textit{Only allow read access to the root filesystem}''. This is problematic because it oversimplifies container security. Generic advice like "Only allow read access to the root filesystem" may seem straightforward, but it doesn't account for the specific context or unique configuration of an individual deployment. Such one-size-fits-all recommendations can lead users to believe that simply following these broad guidelines is enough to secure their systems, potentially leaving critical vulnerabilities unaddressed. In reality, effective container security requires a nuanced, tailored approach that considers the particular needs and threat models of each environment. Houdini fills the gap between generic advice and practical security by enabling you to objectively evaluate the effectiveness of your container confinement measures.

\noindent\textbf{Offensive Tools.} The offensive security community has developed a range of specialized tools aimed at identifying and exploiting vulnerabilities within containerized environments. Containers, while providing strong isolation, are not immune to attacks, and these tools focus on testing the robustness of container defenses. CDK\footnote{\url{https://github.com/cdk-team/CDK}} and DEEPCE\footnote{\url{https://github.com/stealthcopter/deepce}} are two widely used, open-source container penetration testing toolkits that leverage a collection of known exploits to gain persistence, escape the container, and gather sensitive information about the container environment. These tools aim to bypass isolation mechanisms like namespaces, cgroups, and SELinux, often using a variety of aggressive techniques to identify weaknesses in container security. CDK specifically uses any available method to circumvent security measures, providing a broad toolkit to test the limits of a container’s defense. Meanwhile, DEEPCE, developed by stealthcopter, focuses on privilege escalation and container enumeration, attempting to exploit specific weaknesses to achieve container escapes, thereby compromising the host system. In contrast, Houdini takes a more systematic and structured approach to container security testing. Rather than focusing solely on exploiting vulnerabilities, Houdini allows for the direct comparison of defensive techniques by clearly defining the testing environment, outlining the exploit steps, and documenting the expected outputs.

\noindent\textbf{Virtual Machine Escapes} VM escapes occur when an attacker successfully breaks out of a virtualized confined environment, gaining unauthorized access to the host system. Given that hypervisors manage multiple VMs on shared hardware, such escapes can lead to host compromise, cross-VM attacks, and lateral movement within cloud environments. One of the most notable VM escape vulnerabilities is VENOM (CVE-2015-3456) \cite{abdoul2018novel}, which exploited a flaw in the virtual floppy disk controller (FDC) implementation in QEMU, affecting platforms like Xen, KVM, and VirtualBox. Other significant VM escape vulnerabilities include Cloudburst (2009), which exploited VMware's virtual graphics driver; CVE-2018-6981 and CVE-2018-6982, which targeted VMware ESXi, Workstation, and Fusion \cite{vmware2018escape}; and multiple VirtualBox vulnerabilities (2017) enabling guest-to-host code execution \cite{virtualbox2017bugs}. Additionally, side-channel attacks like MDS/ZombieLoad (2019) demonstrated how speculative execution flaws could be leveraged across VMs \cite{mds2019attack}. More recent research has uncovered weaknesses in encrypted virtualization, such as the SEVered attack (2018), which extracted plaintext memory from AMD SEV-protected VMs \cite{severed2018}, and the Heckler attack (2024), which manipulated non-timer interrupts to compromise confidential VMs in AMD SEV-SNP and Intel TDX \cite{heckler2024}.

\noindent\textbf{Sandbox Escapes} Sandboxing is a critical security mechanism for restricting the execution of untrusted code, commonly used in browsers, mobile apps, and security tools. A sandbox escape occurs when an attacker bypasses these restrictions, executing arbitrary code with higher privileges and potentially leading to full system compromise. Modern web browsers heavily rely on sandboxing to isolate JavaScript execution from the underlying OS. However, several high-profile vulnerabilities have enabled sandbox escapes. For example, CVE-2019-5786 \cite{chromezeroday}, a use-after-free vulnerability in the FileReader API of Google Chrome, allowed attackers to escape Chrome's sandbox and execute code on the host. Other notable sandbox escape vulnerabilities include CVE-2022-26706, which exploited a flaw in macOS's App Sandbox to execute code outside of restricted permissions \cite{macossandboxescape}; CVE-2023-36719, a stack corruption vulnerability in Windows OS that enabled escape from the Chromium sandbox \cite{chromiumsandboxescape}; and CVE-2024-4040, which was actively exploited to break out of CrushFTP’s Virtual File System (VFS) \cite{crushftpsandboxescape}. Additionally, vulnerabilities in Judge0, an online code execution system, exposed systems to full takeover by allowing attackers to escape its sandbox \cite{judge0sandboxescape}. Research has also uncovered critical flaws in vm2, a JavaScript sandbox library, which could be exploited to bypass its security restrictions and gain remote code execution \cite{vm2sandboxescape}.